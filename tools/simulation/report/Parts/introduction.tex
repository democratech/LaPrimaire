\documentclass[conference]{IEEEtran}
\newcommand*{\rootPath}{../}
\usepackage{url}

% math and cs
% \usepackage[]{algorithm2e}
\usepackage[linesnumbered,lined,boxed,commentsnumbered]{algorithm2e}
\usepackage{amsmath}
\usepackage{amssymb}
\usepackage{mathrsfs}
\newtheorem{definition}{Definition}
\newtheorem{proposition}{Proposition}
\newenvironment{proof}[1][Proof]{\begin{trivlist}
  \item[\hskip \labelsep {\bfseries #1}]}{\end{trivlist}}
  % \newenvironment{definition}[1][Definition]{\begin{trivlist}
  % \item[\hskip \labelsep {\bfseries #1}]}{\end{trivlist}}
  % \newenvironment{example}[1][Example]{\begin{trivlist}
  % \item[\hskip \labelsep {\bfseries #1}]}{\end{trivlist}}
  % \newenvironment{remark}[1][Remark]{\begin{trivlist}
  % \item[\hskip \labelsep {\bfseries #1}]}{\end{trivlist}}

% style
\usepackage{booktabs}
\usepackage{multirow}
\usepackage{lipsum}
\usepackage{todonotes}
\usepackage{standalone}
\usepackage{import}
\usepackage{url}
%\Urlmuskip=0mu plus 1mu %bug url

% graph
\usepackage{graphicx}
\usepackage[outdir=./]{epstopdf}
\usepackage[labelformat=simple]{subcaption}
\usepackage{array}
%\usepackage[colorinlistoftodos]{todonotes}
\newcommand{\HRule}{\rule{\linewidth}{0.5mm}}



\DeclareCaptionLabelSeparator{periodspace}{.\quad}
\captionsetup{font=footnotesize,labelsep=periodspace,singlelinecheck=false}
\captionsetup[sub]{font=footnotesize,singlelinecheck=true}


\usepackage[english,american]{babel}



\usepackage[capitalise,nameinlink]{cleveref}
%Nice formats for \cref
\crefname{section}{Sect.}{Sect.}
\Crefname{section}{Section}{Sections}
\crefname{figure}{Fig.}{Fig.}
\Crefname{figure}{Figure}{Figures}

\usepackage{xspace}
%\newcommand{\eg}{e.\,g.\xspace}
%\newcommand{\ie}{i.\,e.\xspace}
\newcommand{\eg}{e.\,g.,\ }
\newcommand{\ie}{i.\,e.,\ }


\renewcommand\thesubfigure{(\alph{subfigure})}


%invert table
\usepackage{collcell}
\usepackage{datatool}
\usepackage{environ}


\standalonetrue

\begin{document}

%%=============================================================================



\section{Introduction}
\label{sec:intro}


LaPrimaire.org est une solution citoyenne pour r\'epondre \'a un manque de repr\'esentativit\'e de la soci\'et\'e fran\c{c}aise. En effet, plus de 9 fran\c{c}ais sur 10 ne font plus confiance dans la politique. Les scandales touchent les politiciens de chacun des partis traditionnels. Le nombre de partisans est d'ailleurs en baisse continu.


Notre m\'ethode est de pousser au renouvellement de la classe politique. Pour cela, nous organisons une primaire pr\'esidentielle pour les \'elections de 2017 en France dans laquelle chaque citoyen fran\c{c}ais peut candidater. L'\'elu(e) recevra un soutien financier, l\'egal et les 500 signatures d'\'elus provenant de 30 d\'epartements diff\'erents avec 50 \'elus maximum par d\'epartement.

Pour faciliter l'apparition de nouveaux repr\'esentants, la Primaire repose sur plusieurs principes, comme indiqu\'es dans son manifeste~\cite{manifeste}. Elle n'est pas un parti politique ; elle n'a pas de programme politique. Un candidat n'est pas jug\'e sur sa notori\'et\'e mais sur ses id\'ees.

Le syst\`eme de vote actuellement utilis\'e pour les \'elections pr\'esidentielles, appel\'e \emph{vote majoritaire}, va au contraire de ce dernier principe : une personnalit\'e publique aura plus de facilit\'e qu'un citoyen ordinaire pour rassembler plus de voix gr\^ace \'a sa pr\'esence dans les r\'eseaux sociaux ou les m\'edias. Le vote majoritaire souffre d'autres inconv\'enients que nous pr\'esentons dans la \cref{sec:mv}. 

C'est pour cela que nous nous sommes tournons vers un nouveau syst\`eme de vote, appel\'e jugement majoritaire (JM). Le principe de ce syst\`eme est de juger chaque candidat selon un bar\^eme de mentions: tr\`es bien, bien, assez bien, correct et d\'evaforable. Le gagnant d'une \'election est alors celui qui rassemble les meilleurs mentions dans le sens de la m\'edianne.  Le JM a \'et\'e propos\'e pour la premi\'ere fois par deux chercheurs fran\c{c}ais, Michel Balanski et Rida Lakari \cite{mj}.  Il a \'et\'e test\'e lors des \'elections pr\'esidentielles de 2012 par le \emph{think tank} Terra Nova \cite{terra-nova}. Dans la \cref{sec:mj}, nous r\'esumons ses avantages et inconv\'enients par rapport \`a d'autres syst\`emes de vote. 

Si ce syst\`eme permet d'am\'eliorer la repr\'esentativit\'e du vote, il n'est pas con\c{c}u pour fonctionner avec un grand nombre de candidats. Tous les \'electeurs ne peuvent pas \'etudier de mani\`ere approfondie plus de 100 candidats. Par soucis pratique, les candidats ayant de plus de notori\'et\'e seraient avantag\'es. C'est pourquoi, nous avons deriv\'e le JM pour l'adapter \`a un grand nombre de candidats. Plus concr\`etement, chaque \'electeur re\c{c}oit un lot de 10 candidats \`a juger. Dans la \cref{sec:laprimaire}, nous montrons que comment notre proc\'ed\'e pour construire un lot permet d'assurer l'\'equir\'epartie entre les candidats. Gr\^ace \`a une simulation bas\'ee sur les r\'esultats de Terra Nova \cite{terra-nova}, nous avons \'etabli le nombre minimum d'\'electeurs n\'ecessaire \`a rendre le vote fiable.



\end{document}


